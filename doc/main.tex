\documentclass[12pt]{article}

\usepackage{fullpage}
\usepackage{multicol,multirow}
\usepackage{tabularx}
\usepackage{ulem}
\usepackage[utf8]{inputenc}
\usepackage[russian]{babel}
\usepackage{pgfplots}
\usepackage{enumitem}
\usepackage{listings}

% Оригиналный шаблон: http://k806.ru/dalabs/da-report-template-2012.tex

\begin{document}

\section*{Курсовая работа по курсу дискретного анализа: diff}

Выполнил студент группы 08-308 МАИ \textit{Попов Николай}.

\section{Условие}


 Даны две строки со словами. Необходимо найти наибольшую общую подпоследовательность слов. Обратите внимание на ограничения по памяти, стандартное решение при помощи динамического программирования не подойдет.


\section{Метод решения}

Идея в основе этого алгоритма проста: если разделить входную последовательность $x=x_1 x_2 ... x_m$ на две произвольные части по любому граничному индексу $i$ на $x_b=x_1 x_2 ... x_i$ и $xe= x_i+1 x_i+2 ... x_m$, то найдется способ аналогичного разбиения y (найдется такой индекс $j$, разбивающий $y$ на $y_b=y_1 y_2 ... y_j$ и $y_e= y_j+1 y_j+2 ... y_n$), такой, что $LCS(x,y) = LCS(x_b,y_b) + LCS(x_e,y_e)$.  

\\

Пока в последовательности x есть элементы, мы делим x пополам, находим подходящее разбиение для y и возвращаем сумму рекурсивных вызовов для пар последовательностей $(x_b,y_b)$ и $(x_e,y_e)$. Заметим, что в тривиальном случае, если x состоит из одного элемента и встречается в y мы просто возвращаем последовательность из этого единственного элемента $x$.


\section{Описание программы}

Программа реализует алгоритм нахождения наибольшей общей подпоследовательности (НОП) слов, используя алгоритм Хиршберга. Этот алгоритм является эффективным с точки зрения использования памяти методом для нахождения НОП, так как он требует линейного объема памяти относительно длины последовательностей.

\subsection{Основные компоненты программы}

\begin{itemize}
    \item \textbf{Функция \texttt{lcs\_length}}: Вычисляет длину НОП для двух векторов строк, которые представляют собой входные последовательности слов. Возвращает вектор целых чисел, где каждый элемент указывает максимальную длину НОП до данного индекса во второй последовательности.
    \item \textbf{Функция \texttt{LCS\_HIRSHBERG}}: Реализует алгоритм Хиршберга для определения НОП между двумя последовательностями слов. Эта функция рекурсивно делит каждую последовательность на две части, вычисляет НОП для каждой пары частей, и комбинирует эти результаты.
    \item \textbf{Функция \texttt{readWords}}: Читает входные данные из стандартного ввода и преобразует их в вектор строк. Каждая строка интерпретируется как отдельное слово последовательности.
    \item \textbf{Главная функция \texttt{main}}: Контролирует выполнение программы, включая считывание входных данных, вызов функции \texttt{LCS\_HIRSHBERG} для нахождения НОП и вывод результатов на стандартный вывод.
\end{itemize}

\subsection{Принцип работы программы}

\begin{enumerate}
    \item Считываются две последовательности слов из стандартного ввода.
    \item С использованием алгоритма Хиршберга вычисляется НОП.
    \item Размер НОП и слова, составляющие НОП, выводятся на стандартный вывод.
\end{enumerate}

Программа предназначена для демонстрации использования алгоритма Хиршберга в задаче нахождения НОП между двумя последовательностями слов. Она эффективно управляет памятью и подходит для работы с большими объемами данных.






\section{Тест производительности}


\begin{tikzpicture}
\begin{axis}[
title={Зависимость времени выполнения от размера входных данных},
xlabel={$n = m$},
ylabel={Время выполнения (мс)},
xmin=0, xmax=50000,
ymin=0, ymax=260000,
xtick={0,10000,20000,30000,40000,50000},
ytick={0,50000,100000,150000,200000,250000},
legend pos=north west,
ymajorgrids=true,
grid style=dashed,
]

\addplot[x
color=blue,
mark=square,
]
coordinates {
(0,0)(250,12)(500,42)(1000,120)(2500,679)(5000,2632)(7500,5898)(10000,10324)(12500,16208)(15000,23392)(17500,31486)(20000,41012)(25000, 63791)(27500, 77212)(35000, 125679)(40000, 143633)(45000, 181587)(50000,254547)
};
\legend{LCS}
\end{axis}
\end{tikzpicture}

Исходя из новых результатов тестирования, можно увидеть, что сложность описывается зависимостью $O(nm)$, где 
$n$, $m$ — количество слов в тексте.


\section{Выводы}

В ходе разработки и тестирования программы, реализующей алгоритм Хиршберга для нахождения наибольшей общей подпоследовательности слов, были достигнуты следующие результаты:

\begin{enumerate}
    \item Подтверждена высокая эффективность алгоритма Хиршберга в задачах, требующих оптимизации использования памяти. Реализация алгоритма показала, что он способен обрабатывать большие объемы данных, сохраняя при этом линейную сложность по памяти.
    \item Разработка и тестирование программы обеспечили ценный опыт в реализации и оптимизации алгоритмов для обработки и анализа текстовых данных, подчеркнув важность выбора правильных алгоритмических и структурных решений.
\end{enumerate}

Опыт, полученный в ходе работы над проектом, подчеркивает значимость эффективного управления ресурсами и оптимизации алгоритмов для работы с большими объемами данных. Реализация алгоритма поискка наибольшей общей подподследовательсти при помощи метода Хиршберга демонстрирует, как с помощью различных эвристик можно достигнуть значительного улучшения производительности и эффективности в вычислительных задачах.

\end{document}
